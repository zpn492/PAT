\documentclass[10pt]{article}
\usepackage{listings}
\usepackage{fancyref}
\usepackage{url}
\usepackage{amsmath}


\frefformat{vario}{\fancyrefseclabelprefix}{%
      \frefsecname\fancyrefdefaultspacing#1}
\Frefformat{vario}{\fancyrefseclabelprefix}{%
      \Frefsecname\fancyrefdefaultspacing#1}

\begin{document}

\title{Stochastic Processes in Life Insurance \\
\large Assignment 2}
\author{Benjamin Brandt Ohrt, zpn492}
\maketitle

\section{Introduction}
This is the second of to two assignments in the course Stochastic Processes in Life Insurance at Copenhagen University. The course professor is Jesper Lund Pedersen. The course is held in blok 1, 2019.

\section{Problem 1}

This problem consider the "Standard multi-state policy" described in Theorem 16.2(and Theorem 16.4), that is, we consider a payment process B(t) driven by a finite state Markov process Z(t) given by.

\begin{equation}
dB(t) = b^{Z(t)}(t)dt + \sum_k b^{Z(t-)}(t)dN^k(t)
\end{equation}
Where $b^j(t)$ and $b^{jk}(t)$ are continuous functions. Note that there are no lump sum payments in the a state, that is, $\triangle B^j(t) = 0$. Moreover, we assume that the transition intensities $\mu^{jk}(t)$ also are continuous.

\subsection{Question 1.A}

\begin{equation}
\begin{split}
m^{(2)}(t) 	& = E[Y^2(0,n)|F(t)] 
\end{split}
\end{equation}

By 14.4, we have that m(t) is on the form described below.
We know, that it's an absolute continuous function, and by definition 2.10, 2. case and definition 3.8, and there is an integrable function with respect m(t) which is predictable and therefore adapted to filtration, thus a martingale.
\begin{equation}
\begin{split}
m(t) 	& = \int_0^t {e^{-\int_0^s {r(v) dv}}dB(s)  + e^{-\int_0^t {r(v) dv}}} V(t)
\end{split}
\end{equation}
Use that $(a+b)^2 = 2ab + a^2 + b^2$ and $a^m * a^n = a^{m+n}$
\begin{equation}
\begin{split}
m^{(2)}(t) & = (\int_0^t {e^{-\int_0^s {r(v) dv}}dB(s)  +  e^{-\int_0^t {r(v) dv}}} V(t))^2 \\
 		& =  2 \int_0^t {e^{-\int_0^s {r(v) dv}}dB(s) e^{-\int_0^t {r(v) dv}}} V(t) \\
& + (\int_0^t {e^{-\int_0^s {r(v) dv}}dB(s)})^2 \\
& + (e^{-\int_0^t {r(v) dv}} V(t))^2  \\
 		& = 2 \int_0^t {e^{-\int_0^s {r(v) dv}}dB(s) e^{-\int_0^t {r(v) dv}}} V(t) \\
& + (\int_0^t {e^{-\int_0^s {r(v) dv}}dB(s)})^2 \\
& + e^{-\int_0^t {r(v) dv}} e^{-\int_0^t {r(v) dv}} (V(t) V(t)) \\
 		& = 2 \int_0^t {e^{-\int_0^s {r(v) dv}}dB(s) e^{-\int_0^t {r(v) dv}}} V(t) \\
& + (\int_0^t {e^{-\int_0^s {r(v) dv}}dB(s)})^2 \\
& + e^{-2\int_0^t {r(v) dv}} V^2(t)
\end{split}
\end{equation}

\subsection{Question 1.B}
\begin{equation}
\begin{split}
m^{(2)}(t) =
& 2 \int_0^t {e^{-\int_0^s {r(v) dv}}dB(s) e^{-\int_0^t {r(v) dv}}} V(t) \\
& + (\int_0^t {e^{-\int_0^s {r(v) dv}}dB(s)})^2 \\
& + e^{-2\int_0^t {r(v) dv}} V^2(t)
\end{split}
\end{equation}

Divide the above equation in three, and solve the equations with integration by parts $d(f(t)g(t)) = df(t)g(t) + f(t-)dg(t)$. Use that $\int_0^t e^{-\int_0^s} a(s)$ can be written as $e^{-\int_0^t} \int_0^t a(s)$, $e^{-\int_0^t} (\int_0^{t-} a(s) + \int_{t-}^t a(s))$ and $\int_{t-}^t a(s) = a(t) - a(t-) = \triangle a(t)$
\begin{equation}
\begin{split}
d(2 \int_0^t {e^{-\int_0^s {r(v) dv}}dB(s) e^{-\int_0^t {r(v) dv}}} V(t)) \\
= 2e^{-\int_0^t {r(v) dv}}dB(t)e^{-\int_0^t {r(v) dv}}V(t) \\
+ 2 \int_0^{t-} e^{-\int_0^s r(v)dv} dB(s)(-r(t)e^{-\int_0^t r(v) dv}V(t)dt + e^{-\int_0^t r(v) dv} dV(t)) 
\end{split}
\end{equation}

\begin{equation}
\begin{split}
d((\int_0^t {e^{-\int_0^s {r(v) dv}}dB(s)})^2) \\
= (\int_0^t {e^{-\int_0^s {r(v) dv}}dB(s)})e^{-\int_0^t {r(v) dv}}dB(t) \\
+ (\int_0^{t-} {e^{-\int_0^s {r(v) dv}}dB(s)})e^{-\int_0^t {r(v) dv}}dB(t) \\
= (\int_0^{t-} {e^{-\int_0^s {r(v) dv}}dB(s)})e^{-\int_0^t {r(v) dv}}dB(t) \\
+ (e^{-\int_0^t {r(v) dv}}\int_{t-}^{t} {dB(s)})e^{-\int_0^t {r(v) dv}}dB(t) \\
+ (\int_0^{t-} {e^{-\int_0^s {r(v) dv}}dB(s)})e^{-\int_0^t {r(v) dv}}dB(t) \\
= \triangle B(t) e^{-2\int_0^t {r(v) dv}}dB(t) \\
+ 2(\int_0^{t-} {e^{-\int_0^s {r(v) dv}}dB(s)})e^{-\int_0^t {r(v) dv}}dB(t) \\
\end{split}
\end{equation}

\begin{equation}
\begin{split}
d(e^{-2\int_0^t {r(v) dv}} V^{(2)}(t)) \\
= -2r(t)e^{-2\int_0^t {r(v) dv}} V^{(2)}(t) + e^{-2\int_0^t {r(v) dv}} dV^{(2)}(t)
\end{split}
\end{equation}

change $t-$ to $t$ for the dt part and merge all the equations which leads us to the result below.
\begin{equation}
\begin{split}
dm^{(2)}(t) = 2e^{-\int_0^t {r(v) dv}}dB(t)e^{-\int_0^t {r(v) dv}}V(t) \\
+ 2 \int_0^{t} e^{-\int_0^s r(v)dv} dB(s)(-r(t)e^{-\int_0^t r(v) dv}V(t) dt) \\
+ \triangle B(t) e^{-2\int_0^t {r(v) dv}}dB(t) \\
+ 2(\int_0^{t-} {e^{-\int_0^s {r(v) dv}}dB(s)})e^{-\int_0^t {r(v) dv}}dB(t) \\
-2r(t)e^{-2\int_0^t {r(v) dv}} V^{(2)}(t) \\
+ e^{-2\int_0^t {r(v) dv}} dV^{(2)}(t)
\end{split}
\end{equation}

\newpage

\subsection{Question 1.C}
Multiply the result from 1.B with $e^{2\int_0^t r(v)dv}$ on both sides of the equality sign. And reduce the equation.
\begin{equation}
\begin{split}
e^{2\int_0^t r(v)dv}dm^2(t) & = \\
& + e^{2\int_0^t r(v)dv}2(\int_0^{t-} e^{-\int_0^s {r(v) dv}}dB(s))e^{-\int_0^t r(v)dv}dB(t) \\
& + \triangle B(t)dB(t) \\
& - 2 r(t) V^2(t)dt \\
& + dV^2(t) \\
& + 2 V(t)dB(t) \\
& - e^{2\int_0^t r(v)dv}2(\int_0^{t-} e^{-\int_0^s {r(v) dv}}dB(s))e^{-\int_0^t r(v)dv}r(t)V(t)dt \\
& + e^{2\int_0^t r(v)dv}2(\int_0^{t-} e^{-\int_0^s {r(v) dv}}dB(s))e^{-\int_0^t r(v)dv}dV(t) \\
				& = \\
& + \triangle B(t)dB(t) \\
& - 2 r(t) V^2(t) \\
& + dV^2(t) \\
& + 2 V(t)dB(t) \\
& + 2(\int_0^{t-} {e^{-\int_0^s {r(v) dv}}dB(s)}) e^{\int_0^t r(v)dv} (dB(t) + dV(t) - r(t)V(t)dt) \\
& = \\
& + (\triangle B(t) + 2V(t))dB(t) \\
& - 2 r(t) V^2(t) \\
& + dV^2(t) \\
& + 2(\int_0^{t-} {e^{-\int_0^s {r(v) dv}}dB(s)}) e^{\int_0^t r(v)dv} (dB(t) + dV(t) - r(t)V(t)dt)
\end{split}
\end{equation}
Note that $\triangle B(t) = \sum_k b^{Z(t-)k}(t)\triangle N^k(t)$ and that $\triangle N^k(t) * b^{Z(t)}(t)$ is zero, because N only counts 1, if in state $k \neq j$. Use that $\sum_k b^{Z(t-)k}(t) = \sum_j \sum_{k \neq j} b^{jk}(t)$. Use that $V^{Z(t)} = V^k$ when $dN^k(t)$, due to the assumption that we can only be in one state at a time.
\begin{equation}
\begin{split}
(\triangle B(t) + 2V(t))dB(t) & = \\
& (\triangle B(t) + 2V^{Z(t)}(t))(b^{Z(t)}(t)dt + \sum_k b^{Z(t-)k}(t)dN^k(t)) \\
& = \\
& \triangle B(t)(b^{Z(t)}(t)dt + \sum_k b^{Z(t-)k}(t)dN^k(t)) \\ & + (2V^{Z(t)}(t)b^{Z(t)}(t)dt + \sum_k 2V^{Z(t)}(t)b^{Z(t-)k}(t)dN^k(t)) \\
& = \\
& \sum_j 1_{\{Z(t-)=j\}} (2V^j(t)b^{j}(t)dt + \sum_{k \neq j} (b^{jk}(t) + 2V^k(t))b^{jk}(t)dN^k(t)) \\
& = \\
& 2 \sum_j 1_{\{Z(t-)=j\}} V^j(t)b^{j}(t)dt + \sum_{k} (b^{Z(t-)k}(t) + 2V^k(t))b^{Z(t-)k}(t)dN^k(t))
\end{split}
\end{equation}
Insert into the equation before.
\begin{equation}
\begin{split}
e^{2\int_0^t r(v)dv}dm^2(t) & = \\
& + dV^2(t) \\
& - 2 r(t) V^2(t)dt \\
& + 2 \sum_j 1_{\{Z(t-)=j\}} V^j(t)b^{j}(t)dt + \sum_{k} (b^{Z(t-)k}(t) + 2V^k(t))b^{Z(t-)k}(t)dN^k(t)) \\
& + 2(\int_0^{t-} {e^{-\int_0^s {r(v) dv}}dB(s)}) e^{\int_0^t r(v)dv} (dB(t) + dV(t) - r(t)V(t)dt)
\end{split}
\end{equation}

\newpage

\subsection{Question 1.D}
Assume $V^{(2)}(t)$ has an intensity function $\lambda^{(2)}(t)$. Show that

\begin{equation}
\begin{split}
\lambda^{(2)}(t) + \sum_j 1_{\{Z(t-) = j\}}(2V^j(t)b^j(t) + \sum_{k \neq j}(b^{jk}(t) + 2V^k(2))b^{jk}(t)\mu^{jk}(t)) = 2r(t)V^{(2)}(t)
\end{split}
\end{equation}
The right-handside of 1.C is a martingale known from 14.4. For the third part on the left-handside of 1.C, we have $\int_0^{t-} {e^{-\int_0^s {r(v) dv}}dB(s)}) e^{\int_0^t r(v)dv} (dB(t) + dV(t) - r(t)V(t)dt)$, $dB(t) + dV(t) - r(t)V(t)dt$ is martingale known from 14.4/14.5. Taking a function on it keeps it a martingale. The part $\int_0^{t-} {e^{-\int_0^s {r(v) dv}}dB(s)} $ is left-continuous, hence predictable.  Subtract a martingale on both sides of the equality sign, sets the right-handside to zero, and removes the third part on the left-handside.
\begin{equation}
\begin{split}
0 & = dV^2(t) \\
& - 2 r(t) V^2(t)dt \\
& + 2 \sum_j 1_{\{Z(t-)=j\}} V^j(t)b^{j}(t)dt + \sum_{k} (b^{Z(t-)k}(t) + 2V^k(t))b^{Z(t-)k}(t)\sum_{k\neq j} \mu^{jk}(t)1_{\{Z(t-)=j\}}) \\
\end{split}
\end{equation}
Now use the assumption $V^{(2)}(t)$ has an intensity function $\lambda^{(2)}(t)$ and by definition 9.14, we have that $\lambda^{(2)}(t)dt = dV^{(2)}(t) = \int_0^t \lambda^{(2)}(s)ds = \Lambda^{(2)}(t)$.
\begin{equation}
\begin{split}
0 & = \Lambda^2(t) \\
& - 2 r(t) V^2(t)dt \\
& + 2 \sum_j 1_{\{Z(t-)=j\}} V^j(t)b^{j}(t)dt + \sum_{k} (b^{Z(t-)k}(t) + 2V^k(t))b^{Z(t-)k}(t)\sum_{k\neq j} \mu^{jk}(t)1_{\{Z(t-)=j\}}) \\
\end{split}
\end{equation}

Furthermore we know that in $\sum_{k} (b^{Z(t-)k}(t) + 2V^k(t))b^{Z(t-)k}(t)dN^k(t)) $ we have $dN^k(t)$ with an intensity process by proposition 12.9 $\sum_{k\neq j} \mu^{jk}(t)1_{\{Z(t-)=j\}}$.
Use that information and solve the equation to something similar to 14.5.
\begin{equation}
\begin{split}
0 & =  \Lambda^2(t) - 2 r(t) V^2(t)dt \\
& + 2 \sum_j 1_{\{Z(t-)=j\}} V^j(t)b^{j}(t)dt + \sum_j (\sum_{k\neq j}1_{\{Z(t-)=j\}} (b^{jk}(t) + 2V^k(t))\sum_{k\neq j} \mu^{jk}(t)b^{jk}(t)1_{\{Z(t-)=j\}})) \\
& = \Lambda^2(t) - 2 r(t) V^2(t)dt \\
& + \sum_j 1_{\{Z(t-)=j\}} (2V^j(t)b^{j}(t)dt + \sum_{k\neq j} (b^{jk}(t) + 2V^k(t)) \mu^{jk}(t)b^{jk}(t))) \\
\end{split}
\end{equation}
\begin{equation}
\begin{split}
2 r(t) V^2(t)dt  =  \Lambda^2(t) + \sum_j 1_{\{Z(t-)=j\}} (2V^j(t)b^{j}(t)dt + \sum_{k\neq j} (b^{jk}(t) + 2V^k(t)) \mu^{jk}(t)b^{jk}(t))) \\
\end{split}
\end{equation}

\subsection{Question 1.E}
By proposition 16.4, page. 55.
\begin{equation}
\begin{split}
V_j^{(2)}(t) & = (2r(t)+\mu(t))V_j^{(2)}(t) - 2b^j(t)V_j(t) \\
& -\sum_{k\neq j}\mu^{jk}(t)\sum_{p=0}^2\binom{2}{p}(b^{jk(2)}(t)V^{k(2-p)}(t)) \\
& = 2r(t)V_j^{(2)}(t) - 2b^j(t)V_j(t) \\
& -\sum_{k\neq j}\mu^{jk}(t)(b^{jk(2)}(t) + b^{jk}(t)V^{k}(t) + V^{k(2)}(t) - V_j^{(2)}(t)) 
\end{split}
\end{equation}

\newpage

\subsection{Question 1.F}
By Theorem 16.2, page 53. 
\begin{equation}
\begin{split}
dV_j(t) = r(t)V_j(t) - b^j(t) - \sum_{k \neq j} \mu^{jk}(t) (b^{jk}(t) + V^k(t) - V^j(t))
\end{split}
\end{equation}
By the assignment.
\begin{equation}
\begin{split}
dV_j^{(2)}(t) & = 
 2r(t)V_j^{(2)}(t) - 2b^j(t)V^j(t) \\
 & - \sum_{k \neq j} \mu^{jk}(t) (b^{jk(2)}(t) + 2b^{jk}V^k(t) + V^{k(2)}(t) - V^{j(2)}(t)) \\ 
\end{split}
\end{equation}
By the assignment $R^{jk}(t) = b_j(t) + V^k - V^j$. And using $(a+b-c)^2 = a^2 + b^2 - c^2 + 2ab -2ac - 2bc$ We get $R^{jk(2)}(t)$ as
\begin{equation}
\begin{split}
R^{jk(2)}(t) = b^{jk(2)}(t) + V^{k(2)}(t) - V^{j(2)}(t) + 2b^{jk}V^k - 2b^{jk}V^j - 2V^kV^j
\end{split}
\end{equation}
\begin{equation}
\begin{split}
Var_j(t) & = V_j^{(2)}(t) - (V_j(t))^2
\end{split}
\end{equation}
Solve the equation and use $R = b_j(t) + V^k(t) - V^j(t)$ and $2V^j(t) = (2b_j (t) + 2V^k(t) - 2b_j (t) - 2V^(t) + 2V^j(t) $
\begin{equation}
\begin{split}
dVar_j(t) 	& = d(V_j^{(2)}(t) - (V_j(t))^2) \\
			& = dV_j^{(2)}(t) - 2V_j(t)dV_j(t) \\
			& = dV_j^{(2)}(t) - 2V_j(t) (r(t)V_j(t) - b^j(t) - \sum_{k \neq j} \mu^{jk}(t) (b^{jk}(t) + V^k(t) - V^j(t))) \\
			& = dV_j^{(2)}(t) - (2r(t)(V_j(t)) ^2 -2 b^j(t)V^j(t) \\ 
			& - \sum_{k \neq j} \mu^{jk}(t) (2(b_ {jk}(t)^2 - b_ {jk}(t)^2) + 4b_ {jk}(t)(V^k(t) - V^k(t)) \\
			& + 2b_ {jk}(t)(2V^j(t) - V^j(t)) + 2(V^{k}(t)^2 - V^{k}(t)^2) - 2V^{j}(t)^2) \\
			& = (2r(t)(V_j^{(2)}(t) - (V_j(t)) ^2) \\ 
			& - \sum_{k \neq j} \mu^{jk}(t) (-2(b_ {jk}(t)^2 - b_ {jk}(t)^2) - 4b_ {jk}(t)(V^k(t) - V^k(t)) \\
			& - 2b_ {jk}(t)(2V^j(t) - V^j(t)) - 2(V^{k}(t)^2 - V^{k}(t)^2) \\
			& + 2V^{j}(t)^2 - 2(2V^k(t)V^j(t) - V^k(t)V^j(t))) + b^{jk(2)}(t) + V^{k(2)}(t) - V^{j(2)}(t) ) \\
			& = 2r(t)Var^{j}(t) - \sum_{k \neq j} \mu^{jk}(t) (-2(b_ {jk}(t)^2 - b_ {jk}(t)^2) - 4b_ {jk}(t)(V^k(t) - V^k(t)) \\
			& - 2b_ {jk}(t)(V^j(t) - V^j(t)) - V^{k}(t)^2 + V^{k}(t)^2 - 2(V^k(t)V^j(t) - V^k(t)V^j(t))) \\
			& + R^{jk(2)}(t) + Var^{k}(t) - Var^{j}(t) ) \\
			& = 2r(t)Var^{j}(t)  - \sum_{k \neq j} \mu^{jk}(t)(R^{jk(2)}(t) + Var^{k}(t) - Var^{j}(t))
\end{split}
\end{equation}
We see, that the equation above looks like Thiele and conclude, that we  can use Norberg 7.43, page 81, saying that $V_j(t) = \int_t^n e^{-\int_t^s(r(v)dv)}\sum_k p_{jk}(t,s (dB(s) + \sum_{k \neq l} b_{kl}(s)\mu_{kl}(s)ds)$ just replace $b_{kl}$ with $R^2$  and observe that $dB(s) = 0$. and $r(s)ds$ is multiplied by two in the equation above.
\begin{equation}
\begin{split}
var_j(t) = \int_t^n {e^{-\int_t^s{2r(v)dv}} \sum_k p^{jk}(t,s)\sum_{k\neq l} \mu^{kl}(s)(R^{kl})^2(s)ds}
\end{split}
\end{equation}

\section{Problem 2}

\subsection{Question 2.A}
By page 63, we have differential equations for $f^j(t)$ and $g^j(t)$
\begin{equation}
\begin{split}
df^j(t) & = r(t)f^j(t) - b^j(t) - g^j(t)c_j(t) 
\\ & - \sum_{k\neq j} \mu^{jk}(t) (bj^{jk}(t) + c^{jk}(t)g^k(t) + f^k(t) - f^j(t)) \\
\end{split}
\end{equation}

Compute f(t) and g(t) in the survival model with two states, alive(0) and dead(1).
\begin{equation}
\begin{split}
df^0(t) & = r(t)f^0(t) - b^0(t) - g^0(t)c_0(t) 
\\ & - \sum_{k\neq j} \mu^{01}(b^{01}(t) + c^{01}(t)g^1(t) + f^1(t) - f^0(t))
\\ c_{01}(t) & = -(b^{01}(t) + V^{1*}(t) - V^{0*}(t))
\end{split}
\end{equation}

\begin{equation}
\begin{split}
df^0(t) & = r(t)f^0(t) - b^0(t) - g^0(t)c_0(t) 
\\ & - \sum_{k\neq j} \mu^{01}(b^{01}(t) + c^{01}(t)g^1(t) + f^1(t) - f^0(t))
\\ & = r(t)f^0(t) - b^0(t) - g^0(t)c_0(t) 
\\ & - \mu^{01}(b^{01}(t) + c^{01}(t)g^1(t) + f^1(t) - f^0(t))
\end{split}
\end{equation}
Assume, that the insurance company which issues the policy has no future liabilities within state of death(1). Why $g^1(t) = 0$ and $f^1(t) = 0$. 

\begin{equation}
\begin{split}
df^0(t) & = r(t)f^0(t) - b^0(t) - g^0(t)c_0(t) - \mu^{01}b^{01}(t) + \mu^{01}f^0(t)
\\ & = (r(t) + \mu(t))f^0(t) - b^0(t) - g^0(t)c_0(t) - \mu^{01}(t)b^{01}(t)
\end{split}
\end{equation}

\begin{equation}
\begin{split}
f^0(t) & = \int_t^n e^{-\int_t^s(r(v)+\mu(v))dv} (b^0(s) + g^0(s)c_0(s) + \mu^{01}(s)b^{01}(s))ds
\end{split}
\end{equation}

Where $f^0(n) = 0$.

\begin{equation}
\begin{split}
dg^j(t) & = (q^j(t) + \sum_{k\neq j}
\mu^{jk}(t) q^{jk}(t)) g^j(t) -q^j(t) \\
& - \sum_{k\neq j}(1-q^{jk}(t))\mu^{jk}(t)(\frac{q^{jk}(t)}{1-q^{jk}(t)}+g^k(t)-g^j(t))
\end{split}
\end{equation}

We know, that transitions $q^{01}(t) = 0$ for all $t$ and $g^1(t) = 0$ for all $t$
\begin{equation}
\begin{split}
dg^0(t) & = (q^0(t) + 
\mu^{01}(t) q^{01}(t)) g^0(t) -q^0(t) \\
& - (1-q^{01}(t))\mu^{01}(t)(\frac{q^{01}(t)}{1-q^{01}(t)}+g^1(t)-g^0(t)) \\
& = (q^0(t) + \mu^{01}(t) q^{01}(t) + (1-q^{01}(t))\mu^{01}(t)) g^0(t) -q^0(t) - \mu^{01}(t)q^{01}(t) \\
& = (q^0(t) + \mu^{01}(t))\mu^{01}(t)) g^0(t) -q^0(t) - \mu^{01}(t)q^{01}(t) 
\end{split}
\end{equation}

\begin{equation}
\begin{split}
g^0(t) = \int_t^n e^{-\int_t^s(q^0(v) + \mu^{01}(v))dv} q^0(s)ds
\end{split}
\end{equation}

Where $q^j(t)$ $0 <= q^{jk}(t) < 1$

\newpage

\begin{thebibliography}{9}

\bibitem{jl}
  Jesper Lund Pedersen.
  \textit{Stochastic Processes in Life Insurance: The Dynamic Approach}.
  Department of Mathematical Sciences.

\end{thebibliography}


\end{document}